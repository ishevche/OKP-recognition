\documentclass{article}
\usepackage{graphicx} % Required for inserting images
\usepackage{amsmath}
\usepackage{enumitem}
\usepackage{fontenc}
\usepackage{amssymb}
\usepackage[backend=bibtex,style=numeric,natbib=true,sorting=none]{biblatex}
\addbibresource{bib}

\begin{document}

    Another approach to solving this problem is to check for a specific $k$ whether the given graph is outer-$k$-planar.
    This check can be encoded as a boolean satisfiability problem.
    This problem asks whether it is possible to assign logic values $\textsc{True}$ or $\textsc{False}$ so that all disjunctive clauses are satisfied.
    A disjunctive clause is a single literal or a disjunction of several.
    Literal is either a variable or a negation of a variable, with the former being the positive and the latter the negative literal.

    \subsubsection{SAT Solver}

    Similarly to the ILP algorithm described in~\ref{subsubsec:ILP-def}, this algorithm uses the same ``ordering variables'' $a_{u, v}$ for each pair of vertices $u$ and $v$ that represent the arrangement of the vertices.
    If the boolean variable $a_{u, v}$ is $\textsc{True}$, the vertex $u$ is located before the vertex $v$ and vice versa otherwise.

    Similarly, these variables must account for transitivity, which means that for every triple of vertices $u$, $v$, and $w$ $a_{u, v} \equiv \textsc{True}$ and $a_{v, w} \equiv \textsc{True}$ implies $a_{u, w} \equiv \textsc{True}$.
    This can be written as follows: $a_{u, v} \land a_{v, w} \rightarrow a_{u, w}$.
    Expanding the implication, this transforms into $\overline{a_{u, v} \land a_{v, w}} \lor a_{u, w}$.
    After applying De Morgan's law, we receive $\overline{a_{u, v}} \lor \overline{a_{v, w}} \lor a_{u, w}$, which represents a clause in the SAT problem.

    The next step is to represent the crossing variables $c_{uv, st}$ in terms of the ordering ones for each pair of edges $uv$ and $st$.
    Similarly to the ILP algorithm, we can restrict $c_{uv, st}$ to $\textsc{True}$ if $uv$ and $st$ cross by adding new clauses to the problem.
    The clauses are constructed by making the implications for each of the eight intersecting cases shown in figure~\ref{fig:edge_crossings}, expanding them, and applying De Morgan's law.
    For example, for the case $a_{u,s} = 1$, $a_{s,v} = 1$, and $a_{v,t} = 1$, we start with the logical equation as follows: $a_{u,s} \land a_{s,v} \land a_{v,t} \rightarrow c_{uv, st}$.
    Afterwards, we expand the implication: $\overline{a_{u,s} \land a_{s,v} \land a_{v,t}} \lor c_{uv, st}$.
    Finally, we apply De Morgan's law: $\overline{a_{u,s}} \lor \overline{a_{s,v}} \lor \overline{a_{v,t}} \lor c_{uv, st}$ receiving one of the eight clauses for $c_{uv, st}$.

    % Help
    The last step in the construction of the problem is to count the number of crossings for each edge.
    The goal of this solver is to check whether the number of crossings can be smaller or equal to some constant $k$ for each edge.
    To ensure this, we can build a set of clauses that prevent the problem from being satisfiable if the value $k$ is too small.
    To do so, for every edge $e_0$, we consider all combinations of $e_1, e_2, \dots, e_{k+1}$ for each of which we construct the following clause: $\overline{c_{e_0, e_1}} \lor \overline{c_{e_0, e_2}} \lor \cdots\lor \overline{c_{e_0, e_{k+1}}}$.
    Doing so, we ensure that for each edge $e_0$, no $k + 1$ different edges intersect $e_0$, which effectively means that each edge has at most $k$ crossings if all clauses are satisfied.

    % TODO: implementation details (iterate over all permutations, finding minimal k)
    % TODO: possible optimizations (2k + 2, equivalence instead of implication for c_{...}, eliminating c_{...} at all)

\end{document}