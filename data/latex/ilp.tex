\documentclass{article}
\usepackage{graphicx} % Required for inserting images
\usepackage{amsmath}
\usepackage{enumitem}
\usepackage{amssymb}
\usepackage[backend=bibtex,style=numeric,natbib=true,sorting=none]{biblatex}
\addbibresource{bib}

\begin{document}

    \subsubsection{ILP Solver}\label{subsubsec:ILP-def}

    The result of the integral linear program should be an arrangement of vertices on a line.
    Given the arrangement of the vertices, we can definitively identify whether two edges cross each other.
    Indeed, consider two edges, $uv$ and $st$.
    Without loss of generality, we can assume that $u$ is located before $v$ in the arrangement, $s$ before $t$, and $u$ before $s$.
    Under these assumptions, the edges $uv$ and $st$ cross if and only if $s$ is located before $v$ and $t$ after $v$.
    % Image?

    The arrangement is represented using the so-called ``ordering variables''.
    For every pair of vertices $u$ and $v$, we create a binary variable $a_{u, v}$ that defines the order in which they appear in the arrangement.
    $a_{u, v} = 1$ indicates that $u$ is located before $v$ and vice versa, and $a_{u, v} = 0$ indicates that $v$ is located before $u$ or $u$ and $v$ is the same vertex.

    For these variables, it is crucial to ensure transitivity.
    That is, if $a_{u, v} = 1$ and $a_{v, w} = 1$ which means $u$ is located before $v$ and $v$ is located before $w$, then $u$ must be located before $w$, so the following should hold $a_{u, w} = 1$.
    This can be ensured by the following constraint: $a_{u, w} \geqslant a_{u, v} + a_{v, w} - 1$.
    This constraint will restrict the value of $a_{u, w}$ if and only if both $a_{u, v}$ and $a_{v, w}$ equal $1$.
    Otherwise, the constraint will have no impact on the system at all.

    Having the arrangement of the vertices, we can now deduce for each pair of edges $uv$ and $st$ whether they intersect or not.
    Naturally, there are 24 different arrangements of the vertices, as demonstrated in figure~\ref{fig:edge_crossings}.
    Among them, there are only eight in which the edges intersect.
    Thus, the edge $uv$ crosses the edge $st$ if and only if one of the following holds:
    \begin{itemize}[noitemsep]
        \item $a_{u,s} = 1$, $a_{s,v} = 1$, and $a_{v,t} = 1$
        \item $a_{u,t} = 1$, $a_{t,v} = 1$, and $a_{v,s} = 1$
        \item $a_{v,s} = 1$, $a_{s,u} = 1$, and $a_{u,t} = 1$
        \item $a_{v,t} = 1$, $a_{t,u} = 1$, and $a_{u,s} = 1$
        \item $a_{s,u} = 1$, $a_{u,t} = 1$, and $a_{t,v} = 1$
        \item $a_{t,u} = 1$, $a_{u,s} = 1$, and $a_{s,v} = 1$
        \item $a_{s,v} = 1$, $a_{v,t} = 1$, and $a_{t,u} = 1$
        \item $a_{t,v} = 1$, $a_{v,s} = 1$, and $a_{s,u} = 1$
    \end{itemize}

    To describe this in the linear program, we create a binary variable $c_{uv, st}$ for every pair of edges $uv$ and $st$.
    $c_{uv, st} = 0$ indicates that the edge $uv$ does not cross $st$.
    To ensure the correctness of these values, we impose eight constraints on each variable, one for each case from above.
    For example, considering the case $a_{u,s} = 1$, $a_{s,v} = 1$, and $a_{v,t} = 1$, we impose the following restriction: $c_{uv, st} \geqslant a_{u,s} + a_{s,v} + a_{v,t} - 2$, which ensures that $c_{uv, st}$ equals $1$ whenever the vertices are ordered as $usvt$.
    Making this for each case restricts $c_{uv, st}$ to the value $1$ if the vertices are arranged in one of the eight ``intersecting'' configurations, ensuring that $c_{uv, st} = 0$ is possible only if $uv$ does not cross $st$.

    The algorithm's objective is to minimize the maximal number of crossings per edge.
    This value can be written as follows: $\max_{uv \in E(G)} \sum_{st \in E(G)} c_{uv, st}$.
    Unfortunately, it cannot represent an objective function for a linear program as the $\max$ operation is not linear.
    To solve this, we introduce a new integer variable $k$.
    To ensure that it is equal to the objective value, we impose the following constraint on $k$: $k \geqslant \sum_{st \in E(G)} c_{uv, st}$ for each edge $uv \in E(G)$.

    So, the integer linear program can be described as follows:
    \begin{align*}
        \textbf{minimize}\quad&k\\
        \textbf{subject to}\quad&k \geqslant \sum_{st \in E(G)} c_{uv, st},&&\forall uv \in E(G)\\
        &c_{uv, st} \geqslant a_{u,s} + a_{s,v} + a_{v,t} - 2,&&\forall uv, st \in E(G)\\
        &c_{uv, st} \geqslant a_{u,t} + a_{t,v} + a_{v,s} - 2,&&\forall uv, st \in E(G)\\
        &c_{uv, st} \geqslant a_{v,s} + a_{s,u} + a_{u,t} - 2,&&\forall uv, st \in E(G)\\
        &c_{uv, st} \geqslant a_{v,t} + a_{t,u} + a_{u,s} - 2,&&\forall uv, st \in E(G)\\
        &c_{uv, st} \geqslant a_{s,u} + a_{u,t} + a_{t,v} - 2,&&\forall uv, st \in E(G)\\
        &c_{uv, st} \geqslant a_{t,u} + a_{u,s} + a_{s,v} - 2,&&\forall uv, st \in E(G)\\
        &c_{uv, st} \geqslant a_{s,v} + a_{v,t} + a_{t,u} - 2,&&\forall uv, st \in E(G)\\
        &c_{uv, st} \geqslant a_{t,v} + a_{v,s} + a_{s,u} - 2,&&\forall uv, st \in E(G)\\
        &a_{u, w} \geqslant a_{u, v} + a_{v, w} - 1,&&\forall u, v, w \in V(G)\\
        &c_{uv, st} \in \{0, 1\},&&\forall uv, st \in E(G)\\
        &a_{u, v} \in \{0, 1\},&&\forall u, v \in V(G)\\
    \end{align*}

\end{document}