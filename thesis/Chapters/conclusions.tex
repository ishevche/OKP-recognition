\chapter{Conclusions}\label{ch:conclusions}

In this bachelor thesis, we provide three implementations of algorithms designed to recognise outer \(k\)-planar graphs. Two of them are based on the encoding into the integer linear programming and satisfiability problems and were introduced in this thesis. The last one uses a dynamic programming approach. Additionally, we provide a command line interface to invoke the desired method for a specific graph and receive a minimal circular local crossing number \(k\) alongside the graph's circular drawing where each edge is crossed at most \(k\) times.

We also demonstrated an example of the program's output for a sample graph and the results of the experiments designed to test, evaluate the performance and compare the algorithms with each other. After analysing the results of the experiments, we concluded the following:
\begin{itemize}
    \item Despite the overhead required to perform the biconnected decomposition, doing so significantly improves the performance of all methods.
    \item In the ILP-based algorithm, including constraints for all arrangements of two edges' endpoints significantly worsens its performance. On the other hand, including an extra term in the objective function slightly improves it.
    \item In the SAT-based algorithm, including clauses for all arrangements does not influence performance.
    \item The computational resources required for executing an ILP-based algorithm grow slower compared to both SAT- and DP-based algorithms with the increase of minimal local crossing number.
    \item The computational time required to execute any of the algorithms grows exponentially in both size and minimal local crossing number of the graph.
\end{itemize}

\section{Limitations}

The first and most major limitation of our implementations is the complexity of the underlying algorithm. The exponential dependency of the running time on the parameters of the input graph significantly limits the number of graphs for which using these implementations is practically reasonable.

The current version also limits the number of vertices to \(64\) in the input graph for the DP-based algorithm, as it uses a bitmask for storing subsets of vertices. We implemented this using an integer as a bitmask to lower the memory consumption; hence, this limit might differ for different systems.

\section{Future work}

There are many directions to explore as a future work. First of all, we could combine the designed implementation into a library. By doing so, we make the algorithms accessible from inside the program. Secondly, we could optimise the underlying algorithms discussed in this work.

Another direction of improvement might be developing an algorithm that, for the given embedding of an outer \(k\)-planar graph, draws it using Bézier curves for edges instead of straight lines. Using this approach, we can further improve the interpretability of the drawing.

Finally, we can explore the options for multithreaded implementation of the BP-based algorithm. This direction seems highly promising. By ensuring all threads check configurations with the right sides of the same size, we eliminate the requirement of memory synchronisation. This makes this algorithm embarrassingly parallel, as it requires synchronising all threads less than \(|V(G)|\) times throughout the whole execution.
