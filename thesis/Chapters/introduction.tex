\chapter{Introduction}\label{ch:introduction}

Nowadays, graphs are widely used across different domains and occupations. They prove to be an efficient tool for visualising relational systems. However, as the size of the graphs grows, so does their complexity, which makes understanding their internal structure increasingly difficult.

This problem was acknowledged by the community as early as in the~1980s. One of the key points discussed by the researchers in the field was the importance of reducing the edge crossings for improving visual clarity~\cite{early-few-crossing}. The suspicion that a drawing with fewer edge crossings was easier to comprehend was later confirmed by several experimental studies~\cite{graph-aesthetic-survey}. These studies showed that minimising the crossings in graph representations significantly improves the ability of humans to interpret the structure, particularly when dealing with complex or large graphs.

This led to the exploration of the ideal form for crossing minimising drawings~--~planar ones. However, requiring the drawing to be completely crossing free imposes severe limitations on an underlying graph. While providing a clean structure, these restrictions are often too constraining for many real-world graphs, especially large ones. As a consequence, many processes cannot be represented in a planar manner, so no insights acquired by the decades-long studies are applicable to them.

As a result, the research community expresses a growing interest in exploring graphs whose drawings are beyond planar; see the survey by \citeauthor{beyond-planarity-survey}~\cite{beyond-planarity-survey}. This led to an exploration of alternative approaches for eliminating edge crossings, such as drawings which include only parts of the edges~\cite{break-the-edge,break-the-edge2}, or merge several edges to form a single track~\cite{confluent-drawings}. However, the most popular approach is to consider ``almost'' planar graphs~--~graphs which admit a drawing with a limited number of crossings. Such graphs offer a balance between visual recognisability and structural flexibility while still retaining some of the beneficial structural properties of planar graphs.

The most natural way to impose a limit on the number of crossings is to place an upper bound on the number of crossings per edge for some value \(k\). This class of graphs is among those that have attracted the most interest from the research community lately~\cite{contest}. In this work, however, we went a step further, additionally restricting the vertices to lie on a circle. Drawings that comply with both restrictions are called \emph{outer \(k\)-planar}, and so are the graphs that admit such drawings.


\section{Our contribution}

However, despite all recent advances in this area, all work has been conducted in a theoretical context. Algorithms have not been implemented or tested empirically. Thus, they lack practical validation. In this thesis, we want to address this gap by implementing the most recent recognition algorithm and by introducing two alternative approaches based on Integer Linear Programming (ILP) and Satisfiability (SAT). While it is NP-hard to solve the general integer linear programming problem or to find a satisfying truth assignment for general Boolean formulas, there are very advanced solvers that can help us find the exact solutions for small- to medium-sized instances within an acceptable amount of time. We also evaluate the performance and efficiency of these methods, demonstrating their practical applicability and their limitations.


\section{Structure of the thesis}

\begin{itemize}
    \item \Cref{ch:related-work} discusses the state of the art by reviewing the main pieces of work in the context of planar and almost planar graph drawings. Here, we also discuss the complexity of the problem.
    \item \Cref{ch:proposed-solution} describes three algorithms for recognising outer \(k\)-planar graphs. Additionally, it presents the optimisations we used to improve the performance of the implemented methods.
    \item \Cref{ch:experiments-and-results} describes the experiments that we conducted to evaluate the performance of the implemented algorithms and discusses the results that we obtained.
    \item \Cref{ch:conclusions} summarises the key accomplishments and overviews the results. It also discusses the limitations of our implementations and outlines potential improvements that can be made to overcome them.
\end{itemize}