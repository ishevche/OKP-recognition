\chapter{Related Work}\label{ch:related-work}

Already in the 1980s, researchers in the field of graph drawing acknowledged the importance of reducing the edge crossings for improving visualisation clarity~\cite{early-few-crossing}. The suspicion that a drawing with fewer edge crossings was easier to comprehend was later confirmed by several experimental studies~\cite{graph-aesthetic-survey}. These studies showed that such crossing minimising graph representations significantly improves humans’ ability to interpret the imposed structure, particularly when dealing with complex or large graphs.

The ideal form of crossing minimising drawings -- planar ones, has been focused on by the research community for a long time, with algorithms for recognition of planar graphs presented already in 1974~\cite{linear-p}. However, requiring the drawing to be completely crossing free imposes huge limitations on an underlying graph. While providing a clean structure, these restrictions are often too constraining for many real-world graphs, especially large ones. This has led to a growing interest in exploring graphs close to being planar. Such graphs allow a limited number of crossings while still retaining some of the beneficial structural properties of planar ones.

Despite the potential benefits of studying graphs beyond strict planarity, research in this area has been relatively sparse. In 2018, a survey by Didimo et al.~\cite{beyond-planarity-survey} offered an overview of the state of the research on graph drawing beyond planarity. The survey highlighted key contributions and outlined the challenges within this area of graph theory.

\section{Hardness of relaxation}\label{sec:hardness-of-relaxation}

Most relaxations of strict planarity dramatically increase the complexity of recognising such graphs. So, the general problem of minimising edge crossings in a graph drawing was known to be computationally intractable already in 1983 when Garey and Johnson~\cite{cr_NPC} demonstrated that the \textsc{Crossing Number} problem determining whether a given graph can be drawn with at most \(k\) crossings, is NP-complete. Their proof relies on a reduction from the \textsc{Optimal Linear Arrangement} problem, which is known to be NP-hard.

Minimising the amount of local crossing is also hard. Korzhik and Mohar~\cite{1p-NPH} showed that testing the \(1\)-planarity, recognising whether a graph can be drawn with at most one crossing per edge, is NP-complete. Later, Cabello and Mohar~\cite{one-edge-NPH} showed that testing \(1\)-planarity is still NP-complete even for near-planar graphs, graphs that can be obtained from planar by adding a single edge.

Given the complexity of recognising \(k\)-planarity, researchers considered exploring more restrictive classes of graphs, hoping that imposed limitations could simplify the recognition. One of the considered classes is outer \(k\)-planar graphs, a subclass of \(k\)-planar graphs in which all vertices lie on the outer face of a drawing.

\section{Efficient recognition of some outer \(k\)-planar graph}\label{sec:efficient-recognition-of-some-outer-(k)-planar-graph}

Despite the general recognition problem for outer \(k\)-planar graphs remains NP-hard, efficient algorithms have been developed for specific values of \(k\).

For \(k = 0\), the recognition task simplifies to an outerplanarity test. Regocnition can be accomplished by augmenting the graph with a new vertex connected to all existing ones and testing whether the resulting graph is planar. An alternative approach, described in~\cite{linear-op}, introduces the concept of \(2\)-reducible graphs, which are totally disconnected or can be made totally disconnected by repeated deletion of edges adjacent to a vertex with a degree at most two. The outerplanarity test proposed in this study is based on an algorithm for testing \(2\)-reducibility.

In the case of \(k = 1\), two research groups independently presented linear-time algorithms in~\cite{linear-o1p} and~\cite{linear-o1p_}. Both methodologies utilise the SPQR decomposition of a graph for their testing mechanisms. Notably, the latter solution extends the graph to a maximal outer \(1\)-planar configuration if such a drawing exists, unlike the former, which employs a bottom-up strategy which does not require any transformations to the original graph.

For all other values of \(k\), no research had been conducted until recently when a group of researchers proposed an algorithm for the general case, which we discuss in the next section.

\section{Recognizing general outer \(k\)-planar graphs}\label{sec:recognizing-general-outer-(k)-planar-graphs}

For a given outer \(k\)-planar drawing of a graph, Firman et al.~\cite{triangulations} proposed a method for constructing triangulation with the property that each edge of the triangulation is crossed by at most \(k\) edges of the graph. Since the edges of the triangulation do not necessarily belong to the original graph, they are termed \emph{links} to distinguish them from the graph’s original edges. The construction is done recursively.

Initially, the algorithm selects an edge on the outer face and labels it as the active link. At each recursive step, the active link partitions the graph into two regions: a left part already triangulated and a right part not yet explored. The objective of each step is to triangulate the right portion. To achieve this, a splitting vertex is chosen within the right region, dividing it into two smaller subregions. The splitting vertex is selected so that the two new links, connecting the split vertex with the endpoints of the active link, are each intersected by at most \(k\) edges, which allows including them into the triangulation. The algorithm then recurses, treating each of these newly formed links as the active one.

Later, Kobayashi et al.~\cite{okp} extended this approach to address the recognition problem for outer \(k\)-planar graphs. In contrast to the triangulation task, where the drawing is provided, the recognition problem requires determining whether a given graph admits an outer \(k\)-planar drawing. Although the core idea remains analogous, the absence of a drawing requires the exploration of all possible configurations. Here, each recursive step verifies whether the unexplored right portion of the graph can be drawn as an outer \(k\)-planar graph that is compatible with the left part.

Moreover, instead of relying on recursion, the method utilises a dynamic programming approach. This framework combines solutions of smaller subproblems retrieved from a table to solve larger ones. To populate this table, the algorithm iterates over all possible configurations corresponding to different recursion steps. Several parameters characterise each such configuration. The first parameter is the active link -- a pair of vertices that divides the graph into left and right regions. The second parameter is a set of vertices in the right part, which is not uniquely determined as in the triangulation case. Additionally, the configuration depends on the order in which edges intersect the active link and the number of intersections on the right side for each one of them. These parameters are used to ensure that the drawing of the right portion is compatible with the left part. For each configuration, the algorithm considers all possible ways to split the right region further. For each of them, the method checks whether these splits are compatible with each other and the left part of the drawing.

Using the restriction on the number of edges crossing each link, the authors demonstrated that for a fixed \(k\), the number of possible right subgraphs grows only polynomially with respect to the size of the graph. They preceded by arguing that the overall time complexity of the algorithm is \(2^{O(k \log k)}n^{3k + O(1)}\), showing that the algorithm is efficient for a fixed parameter \(k\).

\section{Our contribution}\label{sec:our-contribution}

A common drawback of the methods described above is the lack of practical validation. Although these algorithms have been analysed and discussed in a theoretical context, they have not been implemented or empirically tested. This thesis addresses this gap by implementing the most recent recognition algorithm and introducing two alternative approaches based on reductions to Integer Linear Programming (ILP) and Satisfiability (SAT). The performance and efficiency of these methods are then evaluated, demonstrating their practical applicability and limitations.